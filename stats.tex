\documentclass[margin=1cm]{standalone}

\usepackage{tikz}
\usepackage{pgfplots}

\pgfplotsset{
	def_axis/.style={
		grid=major,
		grid style={gray!30},
		axis lines = middle,
		axis line style={->},
		% transpose legend=true,
		legend columns=-1,
		legend style={
			at={(0.5,-0.1)},anchor=north,
			/tikz/every even column/.append style={column sep=0.5cm}
		},
		mark size=1pt,
		line width=0.2pt,
	},
	ending/.style={orange,thick},
}

\begin{document}

\begin{tikzpicture}
	\begin{axis}[
		def_axis,
		ylabel={Informationsgewinn/Bit},
		xlabel={MN},
		]
		\input{statsMN.tex}
		\addplot[ending] coordinates {(10,0) (10,10)};
	\end{axis}
\end{tikzpicture}
\begin{tikzpicture}
	\begin{axis}[
		def_axis,
		ylabel={Informationsgewinn/Bit},
		xlabel={MB},
		]
		\input{statsMB.tex}
		\addplot[ending] coordinates {(10,0) (10,4)};
	\end{axis}
\end{tikzpicture}
\\
\begin{tikzpicture}
	\begin{axis}[
		def_axis,
		ylabel={fehlende Information/Bit},
		xlabel={MB/MN},
		width=15cm,
		]
		\input{statsInfo.tex}
		\addplot[ending] coordinates {(20,0) (20,27)};
		\addplot[ending] coordinates {(1,3.3219280948874) (25,3.3219280948874)};
	\end{axis}
\end{tikzpicture}

\end{document}
